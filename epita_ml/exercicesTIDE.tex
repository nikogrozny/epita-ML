\documentclass[a4paper]{article}
\usepackage[UTF8]{inputenc}
\newtheorem{exercice}{exercice}

\begin{document}


\begin{exercice}
Dans le fichier \texttt{C1/data2.csv}, trouvez le total cumulé de développement des provinces contrôlées par Muscovy, Ryazan et Novgorod qui ne produisent pas de céréales ('Grain').
\end{exercice}
\begin{exercice}
Même question mais en appliquant préalablement une fonction qui diminue pour chaque province produisant de la fourrure ('Fur') le développement de 5, sans toutefois pouvoir le descendre en dessous de 3.
\end{exercice}

\begin{exercice}
Dans les données du titanic, utilisez une scatter matrix pour croiser les données age et prix du billet (=fare), tout en gardant la couleur pour évaluer la survie.
\end{exercice}
\begin{exercice}
Ajoutez le genre (=sex) pour rendre la scatter matrix utile. Vous devrez convertir le type de données.
\end{exercice}
\begin{exercice}
Ajoutez un facteur de dispersion aléatoire des points autour de la valeur genre de façon à mieux les visualiser.
\end{exercice}

\begin{exercice}
Importez les données de \texttt{C1/data2.csv}, gardez uniquement les champs adm, dip et mil et entraînez une ACP dessus.
\end{exercice}
\begin{exercice}
Comparez graphiquement les représentations des données utilisant deux axes standards (via une scatter matrix) et celles utilisant les axes de l'ACP.
\end{exercice}

\end{document}