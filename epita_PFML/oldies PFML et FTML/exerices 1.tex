\documentclass[a4paper]{article}


\title{Exercices Pandas / Matplotlib}
\date{}

\begin{document}


\maketitle

\textbf{Question 1}\\

Chargez les données du Titanic. {\'E}liminez les colonnes avec des valeurs manquantes. Produisez la sous-matrice des personnes ayant payé leur billet une valeur inférieure à trois fois la valeur moyenne.\\

Dans toutes les questions suivantes on travaillera à partir de cette sous-matrice uniquement.\\

\textbf{Question 2}\\

Produisez un graphe des champs age et tarif en codant le genre avec une couleur. Que peut-on en déduire concernant le profil respectif des hommes et des femmes ?\\

\textbf{Question 3}\\

Produisez les 6 sous-matrices correspondant aux combinaisons des 3 classes et au 2 genres. Pour chacune de ces sous-matrices, effectuez un subplot affichant l'histogramme du taux de survie en fonction de l'âge. Commentez.\\

\textbf{Question 4}\\

Reprenez maintenant la matrice initiale et essayez de trouver d'autres variables impactant significativement le taux de survie, toutes choses égales par ailleurs.


\end{document}