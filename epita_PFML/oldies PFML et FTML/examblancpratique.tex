\documentclass[a4paper]{article}


\title{Examen blanc pratique}
\date{}

\begin{document}

\maketitle

\textbf{Question 1}\\

Importez les données \texttt{provinces} et ne conservez que les champs \texttt{adm, mil, dip}, \texttt{goods} et \texttt{country}. Ne conservez que les lignes sans NA et appartenant à 5 countries de votre choix. Codez numériquement la variable \texttt{goods} et représentez sur 2 subplots \texttt{goods, adm, mil, dip}.\\

\textbf{Question 2}\\

Pour l'ensemble des couples de variables suivantes, produisez un tableau croisé d'effectifs puis effectuez un test d'indépendance :
\begin{itemize}
	\item \texttt{adm} versus \texttt{dip}
	\item \texttt{adm} versus \texttt{mil}
	\item \texttt{adm} versus \texttt{goods}
\end{itemize}

\textbf{Question 3}\\

Entraînez un arbre de décision pour prédire la variable \texttt{mil}. Représentez-le graphiquement. Produisez la matrice de confusion associée.\\

\textbf{Question 4}\\

Entraînez un svm linéaire \texttt{mil = f(adm,goods)} avec une pénalisation de votre choix. Représentez les droites de séparation et coloriez les points en fonction la valeur réelle sur le scatter plot.

\end{document}