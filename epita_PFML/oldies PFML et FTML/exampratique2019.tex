\documentclass[a4paper]{article}


\title{Examen pratique}
\date{}

\begin{document}

\maketitle

\textbf{Question 1}\\

Importez les données \texttt{titanic} et ne conservez que les champs \texttt{age, fare, sex} (que vous coderez en 0/1), \texttt{pclass} et \texttt{survived}. Visualisez les 2 premiers axes sur un subplot, en coloriant selon \texttt{survived}.\\

\textbf{Question 2}\\

Représentez maintenant les deux autres champs sous la forme d'un tableau croisé d'effectifs (2 $\times$ 3), puis testez l'hypothèse d'indépendance de ces deux variables.\\

\textbf{Question 3}\\

Entraînez un svm linéaire \texttt{survived = f(age,fare)} avec une pénalisation 1, 10 et 100. Représentez les droites de séparation sur le scatter plot.\\

\textbf{Question 4}\\

Entraînez un arbre de décision pour prédire le fait d'avoir survécu, sur les 4 variables et avec les paramètres de votre choix. {\'E}valuez son score.

\end{document}