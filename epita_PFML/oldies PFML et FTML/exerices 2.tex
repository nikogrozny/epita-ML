\documentclass[a4paper]{article}


\title{Exercices Régression}
\date{}

\begin{document}


\maketitle

\textbf{Question 1}\\

Importez les données des provinces d'Europa et ne conservez que les provinces dont le continent est Asia. Dans toute la suite de l'exercice c'est cette sous-matrice qui sera utilisée.\\

\textbf{Question 2}\\

Effectuez une régression linéaire entre les variables adm et dip, puis entre le variables mil et dip. Comparez les deux scores.\\

\textbf{Question 3}\\

Pour chacun des 6 couples de variables parmi adm, mil, dip et dev, produisez dans un subplot :
\begin{itemize}
\item la droite de régression linéaire d'une variable par rapport à l'autre
\item sur le même subplot, le scatter plot des différentes données
\item en codant par une couleur différente chacune des différentes valeurs possibles de 'goods'.
\end{itemize}

\textbf{Question 4}\\

Pour chaque modalité de 'goods' produisez dans un subplot l'histogramme des différentes valeurs de dip.\\

\textbf{Question 5 - Difficile}\\

Utilisez une régression logistique uniquement basée sur les champs adm, dip et mil pour essayez de déterminer si une province produit des pierres précieuses (Gems).


\end{document}